\documentclass[11pt,twoside]{article}

% Text
\usepackage[utf8]{inputenc}

% Graphics
\usepackage{graphicx}
\usepackage{graphics}
\usepackage[dvipsnames, table]{xcolor}
\usepackage{tikz}
\usepackage{microtype}
\usepackage{setspace}

% Geometry
\usepackage{geometry}
 \geometry{
 a4paper,
 left=25mm,
 right=25mm,
 top=30mm,
 bottom=25mm,
 heightrounded,
 headsep=7mm}

% URL
\usepackage{hyperref}

% Tables
\usepackage{tabu}
\usepackage{tabularx}
\usepackage{ltablex}
\usepackage{longtable}
\usepackage{float}

\begin{document}

\begin{center}
\thispagestyle{empty}
\includegraphics[scale=1.25]{Images/PolimiLogo}\\
\vspace{4cm}
\textbf{\Huge{RASD Document}}\\
\vspace{1.5cm}
\textbf{\Large{Version 1.0}}\\
\bigskip \par
by \par
\large{Abdallah Alkhetiar}\\
\large{Daniel Bonardi}\\
\bigskip \bigskip
\large{\today}
\end{center}

\newpage

\setcounter{page}{1}
\begin{center}
\textbf{\Huge{Document details}}
\end{center}
\begin{table}[h!]
\begin{tabu} to \textwidth { |X[0.25,r,p] || X[0.75,l,p]| }
\hline

\textbf{Deliverable:} & RASD\\
\hline
\textbf{Title:} & Requirement Analysis and Verification Document \\
\hline
\textbf{Authors:} & Abdallah Alkhetiar and Daniel Bonardi \\
\hline
\textbf{Version:} & 1.0 \\ 
\hline
\textbf{Date:} & \today \\
\hline
\textbf{Download page:} & \href{https://github.com/Zero3474/AlkhetiarBonardi.git}{\texttt{\color{blue}{https://github.com/Zero3474/AlkhetiarBonardi.git}}} \\
\hline
\textbf{Copyright:} & Copyright © 2025, Abdallah Alkhetiar and Daniel Bonardi \\
& – All rights reserved \\
\hline
\end{tabu}
\end{table}

\newpage

\tableofcontents

\newpage

\section{Introduction}

\subsection{Purpose}
\textbf{Students\&Companies} (S\&C) is an internship university platform that allows matching between students seeking internships and companies offering them. The platform's goal is to facilitate the process of matching students with companies based on student skills, experiences, and interests with the needs and opportunities provided by companies.\\
There are mainly two ways to establish a connection between the two parties: 
\begin{itemize}
	\item \textbf{Recommendation system}: Whenever a new internship becomes available, students compatible with the requirements specified by the company get notified and can decide to apply.
	\item \textbf{Proactive searching}: Students can go through the available internships and apply for the ones they are interested in.
\end{itemize}
When an internship starts through the recommendation system, S\&C collects various kinds of information regarding the quality of the recommendation, for example by asking students and companies to provide feedback and suggestions. \\
The platform helps with the selection process by managing interviews and finalizing choices. It also offers spaces where users can report issues, share concerns, and give updates on the status of an ongoing internship.

\subsubsection{Goals}
\begin{itemize}
\item \textbf{[G1] - Student can create an account} \\
Students need to create their account using the university credentials to verify the status as a student. \\
In addition they have to add their CV, a list of their interests and optionally a list of companies that might interest them.
\item \textbf{[G2] - Company can create an account} \\
Companies need to create their account using a certified email to verify the legitimacy of the company. \\
In addition they can add a detailed description and a list of previous projects that might help them stand out more.
\item \textbf{[G3] - Users can update their accounts} \\
Through a dedicated sections users can update the information stored in their account.
\item \textbf{[G4] - Companies can create internships} \\
A company that has an account on the platform, after logging in using the credentials chosen during the sign up process, can create an internship specifying the skills, experiences and interests required.
\item \textbf{[G5] - Students can view all available internships} \\
A student that has an account on the platform, after logging in using the credentials chosen during the sign up process, can navigate through the available internships created by different companies via a specific search system.
\item \textbf{[G6] - Companies can create forms for students to fill} \\
The application has a dedicated section that allows companies to create forms for the internship they are posting.
\item \textbf{[G7] - Recommendation system} \\
Whenever a company creates an internship, all the students that satisfy the requirements specified in the offer are notified. All the students interested in the internship have limited time to apply. At the end of this period, the applications are closed and the list of interested students are sent to the company, that could decide to select a smaller group of students for the selection process.
\item \textbf{[G8] - Students can apply for the internships} \\
When proactively searching for internships, students can choose the ones that interest them the most and apply for them.
\item \textbf{[G9] - Form evaluation} \\
The company evaluates all the submitted forms for an internship by giving a score to each one.
\item \textbf{[G10] - Ranking candidates} \\
A ranking of the candidates will be created at the end of the interviewing stage based on the scores achieved in the questionnaires.
\item \textbf{[G11] - Users can provide feedbacks} \\
When an internship starts through the recommendation system, both students and companies can provide feedback regarding the quality of the recommendations offered by the system.
\item \textbf{[G12] - Complaints management} \\
During an internship both parties, companies and students, can use an ad hoc function to write complaints regarding the other party.
\end{itemize}

	\subsection{Scope}
	The platform facilitates interaction between two distinct user categories:
\begin{itemize}
	\item Companies (internship providers)
	\item Students (possible candidates)
\end{itemize}
Companies maintain primary responsibility for internship creation and management. Each internship posting must provide this information:
\begin{itemize}
	\item Required skills
	\item Relevant past experiences
	\item Candidate interests
\end{itemize}
During the internship posting, the company can create forms to be filled by the students. In the course of the selection process, performance metrics are collected through the submitted application's form and a final ranking of the candidates is generated. Internship positions will be allocated according to the candidate's rank and the number of available slots. \\
Students can apply for the internship that interest them and if the company accepts they will move to the selection process.

		\subsubsection{World phenomena}
\begin{spacing}{1.5}
\textbf{\textit{WP1}} - Company treats students in a bad manner. \\
\textbf{\textit{WP2}} - Student behaves in a non suitable manner during an internship. \\
\textbf{\textit{WP3}} - Company thinks about hosting an internship and prepares accordingly. \\
\textbf{\textit{WP4}} - Student prepares the resume for the creation of an account. \\
\textbf{\textit{WP5}} - Student does internship.
\end{spacing}
		\subsubsection{Shared phenomena}
\begin{spacing}{1.5}
\textbf{\textit{SP1}} - User registers a new account. \textit{(World Controlled)} \\
\textbf{\textit{SP2}} - User logs in. \textit{(World Controlled)} \\
\textbf{\textit{SP3}} - Students proactively search for an internship. \textit{(World Controlled)}\\
\textbf{\textit{SP4}} - The system shows a list of internships to the student. \textit{(Machine Controlled)} \\
\textbf{\textit{SP5}} - The company inserts data regarding the internship they are creating. \textit{(World Controlled)} \\
\textbf{\textit{SP6}} - The system sends notifications to the users. \textit{(Machine Controlled)} \\
\textbf{\textit{SP7}} - The system terminates the application process at the end of the predetermined period, avoiding further student applications.\textit{(Machine Controlled)} \\
\textbf{\textit{SP8}} - The system sends the list of students that applied for the internship to the company. \textit{(Machine Controlled)} \\
\textbf{\textit{SP9}} - The system sends the forms to the accepted students. \textit{(Machine Controlled)}
\end{spacing}
		\subsubsection{Machine phenomena}
\begin{spacing}{1.5}
\textbf{\textit{MP1}} - The system validates the users identity for the registration. \\
\textbf{\textit{MP2}} - The system checks the credentials of the user for the log in. \\
\textbf{\textit{MP3}} - The system validates data before allowing a user to change its information in the account. \\
\textbf{\textit{MP4}} - The system validates and stores the data regarding a newly created internship. \\
\textbf{\textit{MP5}} - After the creation of an internship, the system identifies compatible students. \\
\textbf{\textit{MP6}} - The system dynamically updates the ranking of candidates after the submission of the score of a student. 
\end{spacing}
	\subsection{Definitions, acronyms and abbreviations}
		\subsubsection{Definitions}
\begin{itemize}
\item \textbf{Users} $\rightarrow$ The users of the applications are both students and a representatives of their company.
\item \textbf{Internship} $\rightarrow$ It refers to projects created by companies to allow students to gain experience in a professional context.
\item \textbf{Posting} $\rightarrow$ It is the act of creating an internship and rendering it public for the students to view.
\item \textbf{Proactive searching} $\rightarrow$ Indicates the action performed by a student to take the initiative and look for the internship that interests them the most.
\item \textbf{Apply for internship} $\rightarrow$ Indicates the action performed by a student to inform the company responsible for that internship of their interest in participating.
\item \textbf{Questionnaires/Forms} $\rightarrow$ They are both used to represent the same concept of a list of questions created by the company to evaluate the candidates.
\item \textbf{Score} $\rightarrow$ The score of a form is the sum of the points achieved in each question. Since each question has a maximum limit of points, then the whole form will have such a limit.
\item \textbf{Feedback} $\rightarrow$ It represent the opinion of the user regarding a specific feature that they tried out.
\item \textbf{Resume} $\rightarrow$ It is the same as referring to a CV.
\item \textbf{Certified account} $\rightarrow$ It refers to an account that when used for sign up or log in action, it redirects the user to the page of the organization.
\item \textbf{Screening} $\rightarrow$ It refers to the process done by a company right after the applications deadline, during which only a part of the candidates are selected based on their CV and interests.
\end{itemize}
		\subsubsection{Acronyms}
\begin{itemize}
\item \textbf{S\&C} $\rightarrow$ Students\&Companies, the name of the application.
\item \textbf{CV} $\rightarrow$ Curriculum Vitae, is a short written summary of a person's career, qualifications, and education.
\item \textbf{API} $\rightarrow$ Application Programming Interface, a software intermediary that allows two applications to talk to each other.
\item \textbf{HR} $\rightarrow$ Human Resource, is a department that manages employees.
\end{itemize}
		\subsubsection{Abbreviations}
\begin{itemize}
\item \textbf{Gn} $\rightarrow$ It is used to list all the goals and the n stands for the $n_{th}$ goal described.
\item \textbf{WPn} $\rightarrow$ It is used to list all the world phenomena and the n stands for the $n_{th}$ phenomena described.
\item \textbf{SPn} $\rightarrow$ It is used to list all the shared phenomena and the n stands for the $n_{th}$ phenomena described.
\item \textbf{MPn} $\rightarrow$ It is used to list all the machine phenomena and the n stands for the $n_{th}$ phenomena described.
\end{itemize}
		
	\subsection{Revision history}	
\begin{itemize}
\item Version 1.0 $\rightarrow$ \textbf{WIP}
\end{itemize}
	\subsection{Reference documents}

	\subsection{Document structure}
\begin{itemize}
\item \textbf{Section 1: Introduction} \\
This section is designed to offer a brief overview of the project, explaining the functionality required for the correct behavior of the application. It also presents a list of definitions, acronyms and abbreviations that could be found in this document.
\item \textbf{Section 2: Overall description} \\
This section presents the structure of the system, with all the possible scenarios and the description of the most important functions that needs to be present in the application.\\
Here is also possible to learn about the domain assumptions that allow the system work as intended.
\item \textbf{Section 3: Specific requirements} \\
This section presents a precise description of every important use case and, through the goal-assumptions-requirements mapping, it emphasize the role of each functionality in reaching each goal requested by the client.\\
It's also possible to understand the various constraints under which the system operates and the software's system attributes.
\item \textbf{Section 4: Formal analysis using Alloy} \\
This section presents a formal description of the system through the Alloy language.
\item \textbf{Section 5: Effort spent} \\
This section presents the number of hours spent per each person of the group for each section.
\end{itemize}
\newpage
\section{Overall description}
	\subsection{Product perspective}
		\subsubsection{Scenarios}
\textbf{\large{Scenario 1}} - Account creation and posting \\
ToSoftware is a big company looking for young talents that could be interested in working full time for them in the future. An HR employee is tasked to design internships and would like to advertise its offers to students. For this purpose he creates an S\&C account using him work email, he also need to provide his personal information and some details about the company. After the creation has been completed he starts creating the post for an internship including details such as the project domain, tasks, required skills/experiences, the applications deadline and any other relevant information. He also creates a form with a series of questions and for each of them sets a maximum score to facilitate the evaluation of the candidates.
\vspace{1\baselineskip} \\
\textbf{\large{Scenario 2}} - Bob wants to do an internship \\
Bob is a third year student at the Genius University and thinks about gaining some work experience, but he does not know from where to start. His friend Sam tells him about S\&C, so he creates an account with his institutional email and also adds his interests and CV. At this point he starts looking for an internship, using the search-bar he inserts some keywords to ensure he finds something he is interested in. After confirming the filters Bob is presented with a list of internships that satisfy the inserted keywords, he selects the one he finds more suitable for himself and applies for it.
\vspace{1\baselineskip} \\
\textbf{\large{Scenario 3}} - Sekiro receives a notification through the recommendation system \\
ToSoftware's S\&C's account manager designs an internship that involves helping in the development of its new game "Light Souls" and posts it on the platform. Sekiro is a student of the Genius University, studying computer science. Some time ago he created an S\&C account and specified his interest in game development, so as soon as ToSoftware posts the internship, Sekiro receives a notification that informs him about the offer. The notification reports all the important information about the internship:
\begin{itemize}
\item[] \textbf{Title} : Student experience as game developer
\item[] \textbf{Description} : We are looking for students willing to have a taste of the working environment at our company while assisting with the development of our new game "Light Souls".
\item[] \textbf{Application deadline} : 22 December, 2024
\item[] \textbf{Location} : Japan
\end{itemize}
\vspace{1\baselineskip}
\textbf{\large{Scenario 4}} - Sekiro applies for ToSoftware's internship \\
Sekiro decides that the internship proposed by ToSoftware is perfect for him so he decides to apply for it. The 22 of December (applications' deadline) ToSoftware receives a list of all the students that applied for their internship. Since there are too many people they decide to pick a limited number of candidates. After some time Sekiro receives a notification saying that he was selected for the internship and had to fill in a form before the 22 of January 2025. As soon as Sekiro submits the form with all the questions answered, ToSoftware's account receives a notification containing the filled form, so an employee gives a score to each answer and saves it. The 22 of January the employee checks the ranking generated by the application and picks the top 10 candidates. Since Sekiro ended up in second place he receives a notification containing all the information necessary to participate in the internship.
\vspace{1\baselineskip} \\
\textbf{\large{Scenario 5}} - ToSoftware shares his opinion regarding the recommendation system \\
ToSoftware's S\&C's account manager, notices that all the students participating in the internship are really talented and hard working so he decides to write a feedback to let S\&C developer know. As soon as he is done writing it, he sends it and the message is stored in S\&C database as a positive feedback.
\vspace{1\baselineskip} \\
\textbf{\large{Scenario 6}} - Sekiro is exploited by the company \\
Since Sekiro is performing incredibly good, ToSoftware's internship manager decides to give him more work to do, even though he know Sekiro was barely managing the few task he was given before. Noticing the increasing amount of work, Sekiro decides to write a complain about the company on the application.

		\subsubsection{Class Diagram}
		
		\subsubsection{State Charts}
	
	\subsection{Product functions}
\begin{itemize}
\item \textbf{Sign-up and login} \\
Both of these functionalities are going to be available to all user categories. \\
The sign-up functionality allows users to create a verified account to use on the platform. Each user will be asked to provide personal data such as name, surname, email, username and password. The user category will be recognized based on the domain of the email used for the sin-up. \\
The login functionality allows user to gain access to an already existing account using the credentials used during sign-up.
\item \textbf{Update account information} \\
This functionality is only available for all user categories. \\
A student can only update a new CV and change its interests, since all the personal information are extracted from the institutional email. \\
A company can only update the description and add old projects.
\item \textbf{Post internship} \\
This functionality is available only for companies. It allows them to create a new post based on an internship they designed. The post must contain a title, a description describing the various activities, the skills required from the candidates to have a chance in being chosen. It also needs all the deadlines like the applications deadline, after which students are not allow to apply for the internship, and the form compilation deadline, that can be added after the applications deadline. And lastly it needs to specify the location an the period of the internship.
\item \textbf{Create a custom form} \\
This functionality is available only for companies. At some point companies need to create a form that allows the to select faster the candidates for the internship. The form is created by adding a list of questions and for each question it needs to be specified a maximum score that will be used to generate the ranking during the selection process. The form can be created at any time, during the posting process or even before the screening.
\item \textbf{Search for specific internships} \\
This functionality is available for all user categories, but is designed to be used mostly by students. Through a search bar users can look for specific internships or use keywords to filter through the one they are seeking.
\item \textbf{Apply for internship} \\
This functionality is only available for students. This action is going to save the student as a candidate for the internship. At the end of the applications deadline, his CV will be sent to the company with to ones of all the other students that applied for the same internship.
\item \textbf{Screen candidates} \\
This functionality is only available for companies. When the applications period ends and the company is sent the list of candidates, the company's employee will be able to select the candidates that he considers suitable for the internship and discard the one that are not.
\item \textbf{Evaluate form} \\
This functionality is only available for companies. When the company receives a submitted form, this functionality allows them to check the answers given by the candidate and grade them. Once all the answers has been graded and the company's employee closes the form, the system evaluates the total score and adds it to the ranking.
\item \textbf{Send feedback} \\
This functionality is only available for all user categories. Using this feature user can share their opinion with S\&C developers regarding the functionalites implemented in the software.
\end{itemize}

	\subsection{User characteristics}
In this section it is provided a more distinct characterization of the two user categories of the platform.
		\subsubsection{Company}
A company is a representative of a company that wants to advertise its internships. After creating an account as a company, the user gains access to all the company reserved functionality such as internship posting, custom form creation, candidates screening and form evaluation.
		\subsubsection{Student}
A student is an individual whose aim is to either improve his skills or make experience in a professional context. To be able to create an account he must be affiliated with a university and own an institutional email.
		\subsubsection{University}
A university is an institution whose aim, in the context of S\&C application, is to monitor the current status of an internship that involves its students and they are responsible for handling complains, especially the ones that might require the interruption of the internship.

	\subsection{Assumptions, dependencies and constraints}
\begin{spacing}{1.5}
\textbf{\textit{A1}} - All users have a valid active email. \\
\textbf{\textit{A2}} - Students upload the correct file in the CV section of their account. \\
\textbf{\textit{A3}} - Companies post correct information about existing internships. \\
\textbf{\textit{A4}} - Companies creates reasonable forms before the applications deadline. \\
\textbf{\textit{A5}} - The recommendation system is reliable enough. \\
\textbf{\textit{A6}} - Companies checks all submitted forms and for each of them grade all questions. \\
\textbf{\textit{A7}} - Feedback and complaints provided by users are truthful and accurate.
\end{spacing}
\newpage
\section{Specific requirements}
	\subsection{External interface requirements}
		\subsubsection{User interfaces}
The platform is presented as a series of different pages, each specialized for one or more features with the same scope. The following ones are considered the most important:
\begin{itemize}
\item \textbf{Login / Sign up page} \\
This page allows the user to choose between one of the two functionalities and insert personal data in different ways based on the option chosen.
\item \textbf{Home Page} \\
In this page users can visualize all the internships that have per posted. In addition there is a search bar that, by entering keywords, allows user to look for specific types of internships.
\item \textbf{Internship creation page} \\
This page is available only for companies, and allow them to fill in all the details required to post about an internship.
\item \textbf{Form design page} \\
This page is available only for companies and it allows them to create custom forms needed to select the candidates that apply for their internship. To create forms, companies write a list of questions an mark them with a maximum score achievable by each one of them. At the end of this process the form can be saved for later use.
\item \textbf{Profile page} \\
This page displays all personal information regarding the user logged in the application. It also allows the user to update different information based on the fact that they are students or companies.
\item \textbf{Report page} \\
This page can allow students to either write feedback to S\&C developers regarding the application's features or to write complaints regarding the status of an ongoing internship.
\end{itemize}
		\subsubsection{Hardware interfaces}
The application does not need to provide any kind of hardware interface since it's main purpose is advertising internships and establish a connection between students and companies.
		\subsubsection{Software interfaces}
The software will need to use some external interface to guarantee its correct behavior. For example, to guaranteed the validity of an institutional email, it needs the university to confirm the address, and the same goes for the company.
		\subsubsection{Communication interfaces}
The platform exploits the internet connection to communicate with the main server in order to store and retrieve all the necessary information and also to perform automatic actions such as sending notification, identifying compatible students for the recommendation system and closing an internship's application's process according to the specified deadline.
	\subsection{Functional requirements}
		\subsubsection{Functional requirements}
		\subsubsection{Requirements mapping}
		\subsubsection{Use cases and diagrams}
		\subsubsection{Traceability matrix}
	\subsection{Performance requirements}
	\subsection{Design constraints}
		\subsubsection{Standards compliance}
		\subsubsection{Hardware limitations}
		\subsubsection{Any other constraints}
	\subsection{Software system attributes}
		\subsubsection{Reliability}
		\subsubsection{Availability}
		\subsubsection{Security}
		\subsubsection{Maintainability}
		\subsubsection{Portability}
		
\section{Formal analysis using Alloy}
	\subsection{Static model}
	\subsection{Dynamic model}

\section{Effort spent}
\begin{itemize}

\item \textbf{Abdallah Alkhetiar}
\begin{table}[H]
\begin{tabu}{| c || l |}
\hline
\textbf{Chapter} & \textbf{Effort} \\
\hline
1 & 10 h \\
\hline
2 & 0 h \\
\hline
3 & 0 h \\
\hline
4 & 0 h \\
\hline
\end{tabu}
\end{table}

\item \textbf{Daniel Bonardi}
\begin{table}[H]
\begin{tabu}{| c || l |}
\hline
\textbf{Chapter} & \textbf{Effort} \\
\hline
1 & 10 h \\
\hline
2 & 0 h \\
\hline
3 & 0 h \\
\hline
4 & 0 h \\
\hline
\end{tabu}
\end{table}

\end{itemize}

\newpage

\section{References}
Wikipedia.

\end{document}