\documentclass[11pt,twoside]{article}

% Graphics
\usepackage{graphicx}
\usepackage{graphics}
\usepackage[dvipsnames, table]{xcolor}
\usepackage{tikz}
\usepackage{microtype}
\usepackage{setspace}

% Geometry
\usepackage{geometry}
 \geometry{
 a4paper,
 left=25mm,
 right=25mm,
 top=30mm,
 bottom=25mm,
 heightrounded,
 headsep=7mm}

% URL
\usepackage{hyperref}

% Tables
\usepackage{tabu}
\usepackage{tabularx}
\usepackage{ltablex}
\usepackage{longtable}
\usepackage{float}

\begin{document}

\begin{center}
\thispagestyle{empty}
\includegraphics[scale=1.25]{Images/PolimiLogo}\\
\vspace{4cm}
\textbf{\Huge{RASD Document}}\\
\vspace{1.5cm}
\textbf{\Large{Version 1.0}}\\
\bigskip \par
by \par
\large{Abdallah Alkhetiar}\\
\large{Daniel Bonardi}\\
\bigskip \bigskip
\large{\today}
\end{center}

\newpage

\setcounter{page}{1}
\begin{center}
\textbf{\Huge{Document details}}
\end{center}
\begin{table}[h!]
\begin{tabu} to \textwidth { |X[0.25,r,p] || X[0.75,l,p]| }
\hline

\textbf{Deliverable:} & RASD\\
\hline
\textbf{Title:} & Requirement Analysis and Verification Document \\
\hline
\textbf{Authors:} & Abdallah Alkhetiar and Daniel Bonardi \\
\hline
\textbf{Version:} & 1.0 \\ 
\hline
\textbf{Date:} & \today \\
\hline
\textbf{Download page:} & \href{https://github.com/Zero3474/AlkhetiarBonardi.git}{\texttt{\color{blue}{https://github.com/Zero3474/AlkhetiarBonardi.git}}} \\
\hline
\textbf{Copyright:} & Copyright © 2025, Abdallah Alkhetiar and Daniel Bonardi \\
& – All rights reserved \\
\hline
\end{tabu}
\end{table}

\newpage

\tableofcontents

\newpage

\section{Introduction}

\subsection{Purpose}
\textbf{Students\&Companies} (S\&C) is an internship university platform that allows matching between students seeking internships and companies offering them. The platform's goal is to facilitate the process of matching students with companies based on student skills, experiences, and interests with the needs and opportunities provided by companies.\\
There are mainly two ways to establish a connection between the two parties: 
\begin{itemize}
	\item \textbf{Recommendation system}: Whenever a new internship becomes available, students compatible with the requirements specified by the company get notified and can decide to apply.
	\item \textbf{Proactive searching}: Students can go through the available internships and apply for the ones they are interested in.
\end{itemize}
When an internship starts through the recommendation system, S\&C collects various kinds of information regarding the quality of the recommendation, for example by asking students and companies to provide feedback and suggestions. \\
The platform helps with the selection process by managing interviews and finalizing choices. It also offers spaces where users can report issues, share concerns, and give updates on the status of an ongoing internship.

\subsubsection{Goals}
\begin{itemize}
\item \textbf{[G1] - Student can create an account} \\
Students need to create their account using the university credentials to verify the status as a student. \\
In addition they have to add their CV, a list of their interests and optionally a list of companies that might interest them.
\item \textbf{[G2] - Company can create an account} \\
Companies need to create their account using a certified email to verify the legitimacy of the company. \\
In addition they can add a detailed description and a list of previous projects that might help them stand out more.
\item \textbf{[G3] - Users can update their accounts} \\
Through a dedicated sections users can update the information stored in their account.
\item \textbf{[G4] - Companies can create internships} \\
A company that has an account on the platform, after logging in using the credentials chosen during the sign up process, can create an internship specifying the skills, experiences and interests required.
\item \textbf{[G5] - Students can view all available internships} \\
A student that has an account on the platform, after logging in using the credentials chosen during the sign up process, can navigate through the available internships created by different companies via a specific search system.
\item \textbf{[G6] - Companies can create forms for students to fill} \\
The application has a dedicated section that allows companies to create forms for the internship they are posting.
\item \textbf{[G7] - Recommendation system} \\
Whenever a company creates an internship, all the students that satisfy the requirements specified in the offer are notified. All the students interested in the internship have limited time to apply. At the end of this period, the applications are closed and the list of interested students are sent to the company, that could decide to select a smaller group of students for the selection process.
\item \textbf{[G8] - Students can apply for the internships} \\
When proactively searching for internships, students can choose the ones that interest them the most and apply for them.
\item \textbf{[G9] - Form evaluation} \\
The company evaluates all the submitted forms for an internship by giving a score to each one.
\item \textbf{[G10] - Ranking candidates} \\
A ranking of the candidates will be created at the end of the interviewing stage based on the scores achieved in the questionnaires.
\item \textbf{[G11] - Users can provide feedbacks} \\
When an internship starts through the recommendation system, both students and companies can provide feedback regarding the quality of the recommendations offered by the system.
\item \textbf{[G12] - Complaints management} \\
During an internship both parties, companies and students, can use an ad hoc function to write complaints regarding the other party.
\end{itemize}

	\subsection{Scope}
	The platform facilitates interaction between two distinct user categories:
\begin{itemize}
	\item Companies (internship providers)
	\item Students (possible candidates)
\end{itemize}
Companies maintain primary responsibility for internship creation and management. Each internship posting must provide this information:
\begin{itemize}
	\item Required skills
	\item Relevant past experiences
	\item Candidate interests
\end{itemize}
During the internship posting, the company can create forms to be filled by the students. In the course of the selection process, performance metrics are collected through the submitted application's form and a final ranking of the candidates is generated. Internship positions will be allocated according to the candidate's rank and the number of available slots. \\
Students can apply for the internship that interest them and if the company accepts they will move to the selection process.

		\subsubsection{World phenomena}
\begin{spacing}{1.5}
\textbf{\textit{WP1}} - Company treats students in a bad manner. \\
\textbf{\textit{WP2}} - Student behaves in a non suitable manner during an internship. \\
\textbf{\textit{WP3}} - Company thinks about hosting an internship and prepares accordingly. \\
\textbf{\textit{WP4}} - Student prepares the resume for the creation of an account. \\
\textbf{\textit{WP5}} - Student does internship.
\end{spacing}
		\subsubsection{Shared phenomena}
\begin{spacing}{1.5}
\textbf{\textit{SP1}} - User registers a new account. \textit{(World Controlled)} \\
\textbf{\textit{SP2}} - User logs in. \textit{(World Controlled)} \\
\textbf{\textit{SP3}} - Students proactively search for an internship. \textit{(World Controlled)}\\
\textbf{\textit{SP4}} - The system shows a list of internships to the student. \textit{(Machine Controlled)} \\
\textbf{\textit{SP5}} - The company inserts data regarding the internship they are creating. \textit{(World Controlled)} \\
\textbf{\textit{SP6}} - The system sends notifications to the users. \textit{(Machine Controlled)} \\
\textbf{\textit{SP7}} - The system terminates the application process at the end of the predetermined period, avoiding further student applications.\textit{(Machine Controlled)} \\
\textbf{\textit{SP8}} - The system sends the list of students that applied for the internship to the company. \textit{(Machine Controlled)} \\
\textbf{\textit{SP9}} - The system sends the forms to the accepted students. \textit{(Machine Controlled)}
\end{spacing}
		\subsubsection{Machine phenomena}
\begin{spacing}{1.5}
\textbf{\textit{MP1}} - The system validates the users identity for the registration. \\
\textbf{\textit{MP2}} - The system checks the credentials of the user for the log in. \\
\textbf{\textit{MP3}} - The system validates data before allowing a user to change its information in the account. \\
\textbf{\textit{MP4}} - The system validates and stores the data regarding a newly created internship. \\
\textbf{\textit{MP5}} - After the creation of an internship, the system identifies compatible students. \\
\textbf{\textit{MP6}} - The system dynamically updates the ranking of candidates after the submission of the score of a student. 
\end{spacing}
	\subsection{Definitions, acronyms and abbreviations}
		\subsubsection{Definitions}
\begin{itemize}
\item \textbf{Users} $\rightarrow$ The users of the applications are both students and a representatives of their company.
\item \textbf{Internship} $\rightarrow$ It refers to projects created by companies to allow students to gain experience in a professional context.
\item \textbf{Posting} $\rightarrow$ It is the act of creating an internship and rendering it public for the students to view.
\item \textbf{Proactive searching} $\rightarrow$ Indicates the action performed by a student to take the initiative and look for the internship that interests them the most.
\item \textbf{Apply for internship} $\rightarrow$ Indicates the action performed by a student to inform the company responsible for that internship of their interest in participating.
\item \textbf{Questionnaires/Forms} $\rightarrow$ They are both used to represent the same concept of a list of questions created by the company to evaluate the candidates.
\item \textbf{Score} $\rightarrow$ The score of a form is the sum of the points achieved in each question. Since each question has a maximum limit of points, then the whole form will have such a limit.
\item \textbf{Feedback} $\rightarrow$ It represent the opinion of the user regarding a specific feature that they tried out.
\item \textbf{Resume} $\rightarrow$ It is the same as referring to a CV.
\item \textbf{Certified account} $\rightarrow$ It refers to an account that when used for sign up or log in action, it redirects the user to the page of the organization.
\end{itemize}
		\subsubsection{Acronyms}
\begin{itemize}
\item \textbf{S\&C} $\rightarrow$ Students\&Companies, the name of the application.
\item \textbf{CV} $\rightarrow$ Curriculum Vitae, is a short written summary of a person's career, qualifications, and education.
\item \textbf{API} $\rightarrow$ Application Programming Interface, a software intermediary that allows two applications to talk to each other.
\item \textbf{HR} $\rightarrow$ Human Resource, is a department that manages employees.
\end{itemize}
		\subsubsection{Abbreviations}
\begin{itemize}
\item \textbf{Gn} $\rightarrow$ It is used to list all the goals and the n stands for the $n_{th}$ goal described.
\item \textbf{WPn} $\rightarrow$ It is used to list all the world phenomena and the n stands for the $n_{th}$ phenomena described.
\item \textbf{SPn} $\rightarrow$ It is used to list all the shared phenomena and the n stands for the $n_{th}$ phenomena described.
\item \textbf{MPn} $\rightarrow$ It is used to list all the machine phenomena and the n stands for the $n_{th}$ phenomena described.
\end{itemize}
		
	\subsection{Revision history}	
\begin{itemize}
\item Version 1.0 $\rightarrow$ \textbf{WIP}
\end{itemize}
	\subsection{Reference documents}

	\subsection{Document structure}
\begin{itemize}
\item \textbf{Section 1: Introduction} \\
This section is designed to offer a brief overview of the project, explaining the functionality required for the correct behavior of the application. It also presents a list of definitions, acronyms and abbreviations that could be found in this document.
\item \textbf{Section 2: Overall description} \\
This section presents the structure of the system, with all the possible scenarios and the description of the most important functions that needs to be present in the application.\\
Here is also possible to learn about the domain assumptions that allow the system work as intended.
\item \textbf{Section 3: Specific requirements} \\
This section presents a precise description of every important use case and, through the goal-assumptions-requirements mapping, it emphasize the role of each functionality in reaching each goal requested by the client.\\
It's also possible to understand the various constraints under which the system operates and the software's system attributes.
\item \textbf{Section 4: Formal analysis using Alloy} \\
This section presents a formal description of the system through the Alloy language.
\item \textbf{Section 5: Effort spent} \\
This section presents the number of hours spent per each person of the group for each section.
\end{itemize}
\newpage
\section{Overall description}
	\subsection{Product perspective}
		\subsubsection{Scenarios}
\textbf{\large{Scenario 1}} - Account creation and posting \\
ToSoftware is a big company looking for young talents that could be interested in working full time for them in the future. An HR employee is tasked to design internships and would like to advertise its offers to students. For this purpose he creates an S\&C account by entering all the information required and starts creating the post for an internship by describing the role covered. He also creates a form with a series of questions and for each of them sets a maximum score to facilitate the evaluation of candidates. \\
\\
\textbf{\large{Scenario 2}} - Bob wants to do an internship \\
Bob is a third year student at the Genius University and thinks about gaining some work experience, but he does not know from where to start. His friend Sam tells him about S\&C, so he creates an account with his institutional email and also adds his interests and CV. At this point he starts looking for an internship, using the search-bar he inserts some keywords to ensure he finds something he is interested in. After confirming the filters Bob is presented with a list of internships that satisfy the inserted keywords, he selects the one he finds more suitable for himself and applies for it. \\
\\
\textbf{\large{Scenario 3}} - Recommendation system \\
ToSoftware's S\&C's account manager creates an internship that involves helping in the development of its new game "Light Souls" and posts it on the platform. Sekiro is a student of the Genius University, studying computer science. He created an S\&C account and specified his interest in game development. As soon as ToSoftware posts the internship, Sekiro receives a notification that informs him about the offer. The notification reports all the important information about the internship:
\begin{itemize}
\item[] \textbf{Title} : Student experience as game developer
\item[] \textbf{Description} : We are looking for students willing to have a taste of the working environment at our company while assisting with the development of our new game "Light Souls".
\item[] \textbf{Application deadline} : 22 December, 2024
\item[] \textbf{Location} : Japan
\end{itemize}
\textbf{\large{Scenario 4}} - Application process \\
\\
\textbf{\large{Scenario 5}} - Internship application period ends \\
\\
\textbf{\large{Scenario 6}} - Company select suitable candidates \\
\\
\textbf{\large{Scenario 7}} - Student behaves bad during an internship \\
\newpage
		\subsubsection{Class Diagram}
		
		\subsubsection{State Charts}
	
	\subsection{Product functions}
	\subsection{User characteristics}
	\subsection{Assumptions, dependencies and constraints}
	
\section{Specific requirements}
	\subsection{External interface requirements}
		\subsubsection{User interfaces}
		\subsubsection{Hardware interfaces}
		\subsubsection{Software interfaces}
		\subsubsection{Communication interfaces}
	\subsection{Functional requirements}
	\subsection{Performance requirements}
	\subsection{Design constraints}
		\subsubsection{standards compliance}
		\subsubsection{Hardware limitations}
		\subsubsection{Any other constraints}
	\subsection{Software system attributes}
		\subsubsection{Reliability}
		\subsubsection{Availability}
		\subsubsection{Security}
		\subsubsection{Maintainability}
		\subsubsection{Portability}
		
\section{Formal analysis using Alloy}

\section{Effort spent}
\begin{itemize}

\item \textbf{Abdallah Alkhetiar}
\begin{table}[H]
\begin{tabu}{| c || l |}
\hline
\textbf{Chapter} & \textbf{Effort} \\
\hline
1 & 10 h \\
\hline
2 & 0 h \\
\hline
3 & 0 h \\
\hline
4 & 0 h \\
\hline
\end{tabu}
\end{table}

\item \textbf{Daniel Bonardi}
\begin{table}[H]
\begin{tabu}{| c || l |}
\hline
\textbf{Chapter} & \textbf{Effort} \\
\hline
1 & 10 h \\
\hline
2 & 0 h \\
\hline
3 & 0 h \\
\hline
4 & 0 h \\
\hline
\end{tabu}
\end{table}

\end{itemize}

\newpage

\section{References}
Wikipedia.

\end{document}